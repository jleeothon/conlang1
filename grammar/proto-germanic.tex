\documentclass{article}

\title{Abridged Proto-Germanic Grammar}
\author{Johnny E. Lee Othon }
\date{July 2019}

\newcommand\oon{ǫ̂}

\begin{document}

\maketitle

\section{Phonology}

Consonants:

\begin{itemize}
\item {Stops p, b, t, d, k, g}
\item {Fricatives: f, þ, s/z, x/h}
\item {Sonorant: m, n, w, r, l, j}
\end{itemize}

Vowels:

\begin{itemize}
\item{Short: i, u, e, a}
\item{Long: ī (ei), ū, ē, ō}
\item{Glide: eu ai, au, iu}
\end{itemize}

\section{Nouns}

The biggest groups of nouns are a-stems (mostly masculine and neuter), ō stems (mostly feminine), and roots and consonant stems (mostly masculine).

Consonant stems:

\begin{center}
\begin{tabular}{ | l | l l | }
\hline
& Singular & Plural \\ 
\hline
Nominative & fōts? & fōtiz \\ 
Accusative & fōtų  & fōtunz \\  
Genitive   & fōtiz & fōt\oon \\
Dative     & fōti  & fōtumaz \\
\hline
\end{tabular}
\end{center}

a-stems

\begin{center}
\begin{tabular}{ | l | l l l l | }
\hline
& \multicolumn{2}{|c}{Masculine} & \multicolumn{2}{c|}{Neuter} \\
& Singular & Plural & Singular & Plural \\ 
\hline
Nominative & wulfaz & wulfōz/ōs & juka & jukō \\ 
Accusative & wulfą & wulfanz & juką & jukō \\
Genitive & wulfas/is & wulfǫ & jukas/is & juk\oon \\
Dative & wulfai & wulfamaz & jukai & jukamaz \\
\hline
\end{tabular}
\end{center}

ō-stems (feminine stems)

\begin{center}
\begin{tabular}{ | l | l l | }
\hline
& Singular & Plural \\
\hline
Nominative & gebō & gebôz \\
Accusative & gebǭ & gebōz \\
Genitive & gebōz & geb\oon \\
Dative & gebōi & gebōmaz \\
\hline
\end{tabular}
\end{center}

\section{Verbs}

\subsection{Strong verbs}

In a nutshell:

\begin{itemize}
\item{part 1 is the infinitive, present participle, all present forms;}
\item{part 2 is the singular past of the indicative;}
\item{part 3 is all other conjugated forms incl. past dual/plural;}
\item{part 4 is the past participle.}
\end{itemize}

\begin{center}
\begin{tabular}{ | c | c c c c c c |}
\hline
Class & Part 1 & Part 2 & Part 3 & Part 4 & Example & Followed by \\
\hline
1  & ī  & ai & i  & i  & rīdaną   & \\
2a & eu & au & u  & u  & freusaną & \\
2b & ū  & au & u  & u  & lūkaną   & \\
3a & i  & a  & u  & u  & bindaną  & [mn] + C \\
3b & e  & a  & u  & u  & werþaną  & [lr] + C \\
4  & e  & a  & e  & u  & beraną   & [mnlr] + ¬C \\
5  & e  & a  & e  & e  & etaną    & ¬[mnlr] + C \\
6  & a  & ō  & ō  & a  & bakaną   & ¬([mnlr] + C) \\
\hline
\end{tabular}
\end{center}

Class 7 verbs are "messy" in PGe, and contain reduplication of the root. They are not interesting for this document. In Proto-Northwest-Germanic, they lost reduplication in favour of ablaut.

Classes 7a through 7c are analogous to classes 1 through 3, and complement class 6 i.e. they are followed by a sonorant (j/w/n/m/l/r) and another consonant.

\begin{center}
\begin{tabular}{ | c | c c c c c c | }
\hline
Class & Part 1 & Part 2 & Part 3 & Part 4 & Example & Notes \\
\hline
7a & ai & ē    & ē    & ai & haitanaz  & \\
7b & au & eu   & eu   & au & hlaupanaz & \\
7c & a  & e    & e    & a  & haldanaz  & Followed by [mnlr] \\
7d & ā  & ē    & ē    & ā  & rādaną    & \\
7e & ō  & eu/ē & eu/ē & ō  & grōanaz   & ``eu'' WGe., ``ē'' NGe. \\
\hline
\end{tabular}    
\end{center}

\subsection{Weak verbs}

\subsubsection{Class 1 weak verbs}

Class 1 weak verbs can be distinguished by their suffixes, and the categories of their meanings.

Three class 1 verbs (``suffixless'') have no suffix in the present or past. It has only three members:

\begin{center}
\begin{tabular}{ | l l l | }
\hline
Infinitive & Preterite stem &  Meaning \\
\hline
bringaną & branht- & ``to bring'' \\
brūkaną & brūht- & ``to use'' \\
būaną & būd- & ``to dwell'' \\
\hline
\end{tabular}
\end{center}

\begin{itemize}
\end{itemize}

Five class 1 verbs (``j-present'') have a -j- suffix in the present but no suffix in the preterite. Modern reflexes have umlaut in the present but not in the preterite (``rückumlaut''). It has five members:

\begin{center}
\begin{tabular}{ | l l l |}
\hline
Infinitive & Preterite stem &  Meaning \\
\hline
bugjaną   & buht-  & ``to buy'' \\
sōkijaną  & sōht-  & ``to seek'' \\
þankijaną & þanht- & ``to think'' \\
þunkijaną & þunht- & ``to seem'' \\
wurkijaną & wurht- & ``to work'' \\
\hline
\end{tabular}
\end{center}

The rest of class 1 weak verbs have a -j- suffix both in the present in the past. They can often be causative verbs (e.g. ``to fall'' -> ``to fell''). Some are based on verbal roots, some are based on nominals e.g. ``hailaz'' -> ``hailijaną'' ("to heal").

Some example causatives from strong verbs:

\begin{center}
\begin{tabular}{| c l l l l |}
\hline
Class & Original & Meaning & Causative & Meaning \\
\hline

1   & *līþaną   & to undergo & *laidijaną  & to lead \\
2a  & *beuganą  & to bend    & *baugijaną  & to bend sth. \\
3a  & *brinnaną & to burn    & *brannijaną & to burn sth. \\
3b  & ?        & ?          & ?          & ? \\
4   & ?        & ?          & ?          & ? \\
5c  & *ligjaną  & to lie     & *lagjaną    & to lay \\
5-j & *ligjaną  & to lie     & *lagjaną    & to lay \\
6   & *faraną   & to travel  & *fōrijaną   & to carry or lead \\
7a  & *swaipaną & to swoop   & *swaipijaną & to sweep \\
7b  & *stautaną & to push    & *stautjaną  & to push / to support(?) \\
7c  & *fallaną  & to fall    & *fallijaną  & to fell \\
7d  & *grētaną  & to weep    & *grōtijanaą & to address \\
\hline
\end{tabular}
\end{center}

Some example causatives from nominals:

\begin{center}
\begin{tabular}{| l l l l |}
\hline
Original & Meaning & Causative & Meaning \\
\hline
*hailaz & whole & *hailijaną & heal \\
*blōþą  & blood & *blōþijaną & bleed \\
\hline
\end{tabular}
\end{center}

\subsubsection{Class 2 weak verbs}

Class 2 has a -o- suffix. They are mainly denominatives (i.e. from nouns). E.g. \emph{ *salbōną} (``to anoint'') from \emph{*salbō} (``salve'').

\begin{center}
\begin{tabular}{| l l l l |}
\hline
Noun & Meaning & Infinitive & Meaning \\
\hline
\emph{*bōks}  & book & \emph{*bōkōną} & bleed \\
\emph{*fiskaz}  & book & \emph{*fiskōną} & bleed \\
\emph{*salbō} & salve & \emph{*salbōną} & heal \\
\hline
\end{tabular}
\end{center}

\subsubsection{Class 3 weak verbs}

Class 3 has an -āi- suffix. They are

What is known as "Class III" was actually two separate classes in Proto-Germanic:

A class of verbs with stative semantics (i.e. denoting a state rather than an action), formed with a present suffix that was either *-ai- or *-ja-, and no suffix in the past.

A class of verbs with factitive semantics (i.e. with the meaning "make X" where X is an adjective or noun, e.g. "renew, enslave"), formed with a suffix that was either *-ai- or *-ā-, and a suffix *-a- in the past.

Only four stative verbs survive as Class III verbs in the northern West Germanic languages (i.e. Old English, Old Saxon, Old Frisian and Old Low Franconian):

    *sagjaną "to say"
    *libjaną "to live"
    *habjaną "to hold, have"
    *hugjaną "to think"

However, there are five more verbs that appear as Class III verbs in Old High German, Gothic and/or Old Norse that also have remnants of the stative conjugation in one or more northern West Germanic languages:

    *þagjaną "to be silent"
    *siljaną "to be silent"
    *þuljaną "to endure" (normally Class II þolian in Old English, but cf. archaic umlauted infinitive -þoelġe; Class III in Old Norse þola)
    *fijaną "to hate"
    *hatjaną "to hate" (normally Class II hatian in Old English, but cf. umlauted nominalized present participle hettend "enemy"; Class III in Gothic hatan)    

\subsubsection{Class 4 weak verbs}

Class 4 has a a -na- suffix. They have an inchoative meaning (e.g. ).

\subsection{Preterite-present verbs}

These verbs exhibit vowel alternations in the present, and a dental suffix in the past.

They originate from preterites of strong verbs that started being used as present tenses; and they acquired a new past tense with a dental suffix, like a weak verb.

\begin{center}
\begin{tabular}{|l l l l l l l|}
\hline
Class & Infinitive & Pres. sg. & Pres. pl. & Pret. & Past Ptc. & Meaning \\
\hline
1 & witaną & wait- & wit- & wiss- & wissaz & to know \\
1 & aiganą & āih & āihum & āihta & āihan & to have \\
2 & duganą & daug- & dug- & duht- & duhtaz & to fit \\
3 & kunnaną & kann- & kunn- & kunþ- & kunþaz & to know \\
3 & unnaną & ann- & unn- & unþ- & unþaz & to love \\
3 & þurbaną & þarf- & þurb- & þurft- & þurftaz & to have need of \\
3 & durzaną & dars- & durz- & durst- & durstaz & to dare \\
4 & skulaną & skal- & skul- & skuld- & skuldaz & to have to \\
4 & munaną & man- & mun- & mund- & mundaz & to believe \\
5 & ganuganą & ganah- & ganug- & ganuht- & ganuhtaz & to suffice \\
6 & maganą & mag- & mag- & maht- & mahtaz & to be able \\
6 & ōganą & ag- & ōg- & aht- & ahtaz & to fear \\
6 & mōtaną & mōt- & mōt- & mōs- & mōsaz & to find \\
\hline
\end{tabular}
\end{center}

\end{document}
