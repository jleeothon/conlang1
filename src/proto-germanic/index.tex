\documentclass{article}

\usepackage{fontspec}
\usepackage{multirow}

\setmainfont{Charis SIL}

\title{Reduced Grammar of Proto-Germanic}
\author{Johnny E. Lee Othon }
\date{July 2019}

% Trimoraic nasal o
\newcommand\oon{ǫ̂}

% asterisk
\newcommand\st{\symbol{"002A}}

% Reconstruction
\newcommand\rec[1]{#1}
\newcommand\recem[1]{\emph#1}

\begin{document}

\maketitle

\section{Introduction}

This document contains a summarised grammar of Proto-Germanic, leaving out as many ``details'' as possible, so that it remains useful to the creation of Þeudisk-H.

Caveats to keep in mind when reading this document:

\begin{itemize}
\item subjunctive mood is ignored;
\item dual number is ignored;
\item reconstructions are not marked with an asterisk;
\item consonant stems do not show all possible reconstructed inflections;
\item class 7 strong verbs are analysed as in Proto-West-Germanic and without reduplication;
\item ⟨ei⟩ usually appears as ⟨ī⟩ in other literature.
\end{itemize}

\section{Phonology}

Consonants:

\begin{itemize}
\item {Stops p, b, t, d, k, g}
\item {Fricatives: f, þ, s/z, x/h}
\item {Sonorant: m, n, w, r, l, j}
\end{itemize}

Vowels:

\begin{itemize}
\item{Short: i, u, e, a}
\item{Long: ū, ē, ō}
\item{Glide: ei, eu ai, au, iu}
\end{itemize}

\section{Nouns}

The biggest groups of nouns are a-stems (mostly masculine and neuter), ō stems (mostly feminine), and roots and consonant stems (mostly masculine).

Consonant stems (mostly masculine and neuter):

\begin{center}
\begin{tabular}{ | l | l l | }
\hline
& Singular & Plural \\
\hline
Nominative & \rec{fōt} & \rec{fōtiz} \\
Accusative & \rec{fōtų}  & \rec{fōtunz} \\
Genitive   & \rec{fōtiz} & \rec{fōt\oon} \\
Dative     & \rec{fōti}  & \rec{fōtumaz} \\
\hline
\end{tabular}
\end{center}

a-stems (mostly masculine and neuter):

\begin{center}
\begin{tabular}{ | l | l l l l | }
\hline
& \multicolumn{2}{|c}{Masculine} & \multicolumn{2}{c|}{Neuter} \\
& Singular & Plural & Singular & Plural \\
\hline
Nominative & \rec{wulfaz} & \rec{wulfōs}   & \rec{juka}  & \rec{jukō}    \\
Accusative & \rec{wulfą}  & \rec{wulfanz}  & \rec{juką}  & \rec{jukō}    \\
Genitive   & \rec{wulfas} & \rec{wulfǫ}    & \rec{jukas} & \rec{juk\oon} \\
Dative     & \rec{wulfai} & \rec{wulfamaz} & \rec{jukai} & \rec{jukamaz} \\
\hline
\end{tabular}
\end{center}

ō-stems (mostly feminine):

\begin{center}
\begin{tabular}{ | l | l l | }
\hline
& Singular & Plural \\
\hline
Nominative & \rec{gebō}  & \rec{gebôz} \\
Accusative & \rec{gebǭ}  & \rec{gebōz} \\
Genitive   & \rec{gebōz} & \rec{geb\oon} \\
Dative     & \rec{gebōi} & \rec{gebōmaz} \\
\hline
\end{tabular}
\end{center}

\section{Adjectives}

Most adjectives declined similar to a-stem and ō-stem nouns.


\begin{center}
\begin{tabular}{ | l | l l l l l l | }
\hline
& \multicolumn{3}{|c}{Singular}
& \multicolumn{3}{c|}{Plural} \\
& Masculine & Neuter & Feminine
& Masculine & Neuter & Feminine \\
\hline
Nom. & blidaz & blinda & blindō & blindai & blindō & blindôz \\
Acc. & blindanǫ & blinda & blindǭ & blindanz & blindō & blindôz \\
Gen. & \multicolumn{2}{|c}{blindaz} & blindaizōz & \multicolumn{3}{c|}{blindaiz\oon} \\
Dat. & \multicolumn{2}{|c}{blindammai} & blindaizōi & \multicolumn{3}{c|}{blindaimaz} \\
Instr. & \multicolumn{2}{|c}{blindana} & blindaizō & \multicolumn{3}{c|}{blindaimiz} \\
\hline
\end{tabular}
\end{center}

\section{Verbs}

\subsection{Strong verbs}

The parts of the strong verb paradigm are:

\begin{itemize}
\item{part 1 is the infinitive, present participle, all present forms;}
\item{part 2 is the singular past of the indicative;}
\item{part 3 is all other conjugated forms incl. past dual/plural;}
\item{part 4 is the past participle.}
\end{itemize}

\begin{center}
\begin{tabular}{ | c | c c c c l l |}
\hline
Class & Part 1 & Part 2 & Part 3 & Part 4 & Example & Followed by \\
\hline
1  & ei & ai & i  & i & \rec{reidaną}  & \\
2  & eu & au & u  & u & \rec{freusaną} & \\
3a & i  & a  & u  & u & \rec{bindaną}  & [mn] + C \\
3b & e  & a  & u  & u & \rec{werþaną}  & [lr] + C \\
4  & e  & a  & e  & u & \rec{beraną}   & [mnlr] + ¬C \\
5  & e  & a  & e  & e & \rec{etaną}    & ¬[mnlr] + C \\
6  & a  & ō  & ō  & a & \rec{bakaną}   & ¬([mnlr] + C) \\
\hline
7a & ai & ē    & ē    & ai & haitaną  & \\
7b & au & eu   & eu   & au & hlaupaną & \\
7c & a  & e    & e    & a  & haldaną  & [mnlr] \\
7d & ā  & ē    & ē    & ā  & rādaną   & \\
7e & ō  & eu   & eu   & ō  & hrōpaną  & \\
\hline
\end{tabular}
\end{center}

\subsection{Weak verbs}

\subsubsection{Class 1 weak verbs}

Class 1 weak verbs can be distinguished by their suffixes, and the categories of their meanings.

Three class 1 verbs (``suffixless'') have no suffix in the present or past. It has only three members:

\begin{center}
\begin{tabular}{ | l l l | }
\hline
Infinitive & Preterite stem &  Meaning \\
\hline
bringaną & branht- & to bring \\
brūkaną & brūht- & to use \\
būaną & būd- & to dwell \\
\hline
\end{tabular}
\end{center}

Five class 1 verbs (``j-present'') have a -j- suffix in the present but no suffix in the preterite. Modern reflexes have umlaut in the present but not in the preterite (``rückumlaut''). It has five members:

\begin{center}
\begin{tabular}{ | l l l |}
\hline
Infinitive & Preterite stem &  Meaning \\
\hline
bugjaną   & buht-  & to buy \\
sōkijaną  & sōht-  & to seek \\
þankijaną & þanht- & to think \\
þunkijaną & þunht- & to seem \\
wurkijaną & wurht- & to work \\
\hline
\end{tabular}
\end{center}

The rest of class 1 weak verbs have a -j- suffix both in the present in the past. They can often be causative verbs (e.g. ``to fall'' -> ``to fell''). Some are based on verbal roots, some are based on nominals e.g. ``hailaz'' -> ``hailijaną'' ("to heal").

Some example causatives from strong verbs:

\begin{center}
\begin{tabular}{| c l l l l |}
\hline
Class & Original & Meaning & Causative & Meaning \\
\hline

1   & līþaną   & to undergo     & laidijaną  & to lead \\
2a  & beuganą  & to bend        & baugijaną  & to bend sth. \\
3a  & brinnaną & to be on fire  & brannijaną & to burn sth. \\
3b  & ?        & ?              & ?          & ? \\
4   & ?        & ?              & ?          & ? \\
5c  & ligjaną  & to lie         & lagjaną    & to lay \\
5-j & ligjaną  & to lie         & lagjaną    & to lay \\
6   & faraną   & to travel      & fōrijaną   & to carry or lead \\
7a  & swaipaną & to swoop       & swaipijaną & to sweep \\
7b  & stautaną & to push        & stautjaną  & to push / to support(?) \\
7c  & fallaną  & to fall        & fallijaną  & to fell \\
7d  & grētaną  & to weep        & grōtijanaą & to address \\
\hline
\end{tabular}
\end{center}

When the causative verbs from strong verbs also show these alternations:

\begin{center}
\begin{tabular}{|cc|}
\hline
Alternation & Example \\
\hline
s > z & reisaną > raizijaną \\
þ > d & leiþaną > laidijaną \\
h > g & teuhaną > taugijaną \\
\hline
\end{tabular}
\end{center}

Some example causatives from nominals:

\begin{center}
\begin{tabular}{| l l l l l |}
\hline
Nominal & Meaning & Infinitive & Past Stem & Meaning \\
\hline
hailaz & whole & hailijaną & hailid- & heal \\
blōþą  & blood & blōþijaną & blōþid- & bleed \\
\hline
\end{tabular}
\end{center}

\subsubsection{Class 2 weak verbs}

Class 2 has a -o- suffix. Many of these verbs derive from nouns (denominative). E.g. \textit{ salbōną} (``to anoint'') from \textit{salbō} (``salve'').

They were originally class 1 weak verbs that formed out of nouns ending in -ō, but -j- was contracted between vowels; this resulted in a new class. Other nouns, not ending in -ō started joining this class.

\begin{center}
\begin{tabular}{| l l l l l |}
\hline
Noun & Meaning & Infinitive & Past stem & Meaning \\
\hline
salbō  & salve & salbōną & salbōd- & to heal \\
\hline
fiskaz & fish  & fiskōną & fiskōd- & to fish \\
bōks   & book  & bōkōną  & bōkōd-  & to book \\
\hline
\end{tabular}
\end{center}

\subsubsection{Class 3 weak verbs}

Class 3 has an -ā- suffix, and can be subdivided as:

\begin{itemize}
\item Stative verbs (denoting state):
    \begin{itemize}
    \item present suffix -ai- or -ja-,
    \item no past suffix.
    \end{itemize}
\item Factitive verbs (make sth. x):
    \begin{itemize}
    \item present suffix -ai- or -ā-,
    \item past suffix -a-.
    \end{itemize}
\end{itemize}

In northern West Germanic languages, only the first four stative weak verbs are found:

\begin{center}
\begin{tabular}{| l l l |}
\hline
Infinitive & Past stem & Meaning \\
\hline
sagjaną & sagd- & to say \\
libjaną & libd- & to live \\
habjaną & habd- & to have \\
hugjaną & hugd- & to think \\
\hline
\end{tabular}
\end{center}

Examples of the other class 3 weak verbs are:

\begin{center}
\begin{tabular}{| l l l |}
\hline
Infinitive & preterite stem & Meaning \\
\hline
fūlāną  & fūlad-  & to rot (foul) \\
wakāną  & wakad-  & to awaken \\
þurzāną & þurzad- & to become dry \\
\hline
\end{tabular}
\end{center}

\subsubsection{Class 4 weak verbs}

Class 4 has a a -na- suffix. They have an inchoative meaning i.e. indicate a change of state e.g. \emph{kwiknaną} (to come to live).

\begin{center}
\begin{tabular}{| l l l |}
\hline
Infinitive & preterite stem & Meaning \\
\hline
kwiknaną & kwiknōd- & to become alive \\
wakāną  & wakad-  &  \\
þurzāną & þurzad- & to become dry \\
\hline
\end{tabular}
\end{center}

\subsection{Preterite-present verbs}

These verbs exhibit vowel alternations in the present, and a dental suffix in the past.

They originate from preterites of strong verbs that started being used as present tenses; and they acquired a new past tense with a dental suffix, like a weak verb.

\begin{center}
\begin{tabular}{|l l l l l l l|}
\hline
Class & Infinitive & Pres. sg. & Pres. pl. & Pret. & Past Ptc. & Meaning \\
\hline
1 & witaną & wait- & wit- & wiss- & wissaz & to know \\
1 & aiganą & āih & āihum & āihta & āihan & to have \\
2 & duganą & daug- & dug- & duht- & duhtaz & to fit \\
3 & kunnaną & kann- & kunn- & kunþ- & kunþaz & to know \\
3 & unnaną & ann- & unn- & unþ- & unþaz & to love \\
3 & þurbaną & þarf- & þurb- & þurft- & þurftaz & to have need of \\
3 & durzaną & dars- & durz- & durst- & durstaz & to dare \\
4 & skulaną & skal- & skul- & skuld- & skuldaz & to have to \\
4 & munaną & man- & mun- & mund- & mundaz & to believe \\
5 & ganuganą & ganah- & ganug- & ganuht- & ganuhtaz & to suffice \\
6 & maganą & mag- & mag- & maht- & mahtaz & to be able \\
6 & ōganą & ag- & ōg- & aht- & ahtaz & to fear \\
6 & mōtaną & mōt- & mōt- & mōs- & mōsaz & to find \\
\hline
\end{tabular}
\end{center}

\end{document}
