\documentclass{article}

\usepackage{fontspec}

\setmainfont{Charis SIL}

\title{Þeudisk-H}
\author{Johnny Lee Othon}
\date{July 2019}

\begin{document}

\maketitle

\section{Introduction}

Þeudisk-H is yet another constructed Germanic language, most influenced by current reconstructions of Proto-Germanic, and some West Germanic daughter languages.

Its design philosophy follows these principles:

\begin{itemize}
\item More inflectional morphology than modern West Germanic languages, but less complexity than Proto-Germanic.
\item High degree of symmetry in derivation, conjugation, inflection despite complexity.
\item Freer word order.
\item Mildly pro-drop.
\item Revival of unproductive derivation principles.
\end{itemize}

\section{Phonology and orthography}

The Þeudisk-K alphabet is based on the Latin alphabet. Alternate spelling systems are acceptable but not included in the scope of this document.

\subsection{Vowels}

Þeudisk-K vowels be summarised as:

\begin{itemize}
\item short monophthongs,
\item a weak vowel vowel,
\item long vowels and diphthongs.
\end{itemize}

Vowels may be ``plain'' or ``umlauted''. Plain vowels appear anywhere, and umlauted variants occur in certain derivation paradigms.

Short vowels and orthographic monophthongs:

\begin{center}
\begin{tabular}{cc|cc}
\multicolumn{2}{c}{Short} &
\multicolumn{2}{c}{Long} \\
\hline
i & /ɪ/ & í & /iː/ \\
e & /ɛ/ & é & /eː/ \\
u & /ʊ/ & ú & /uː/ \\
o & /ɔ/ & ó & /oː/ \\
a & /ɑ/ & á & /aː/ \\
ü & /ʏ/ & \H{u} & /yː/ \\
ö & /œ/ & \H{o} & /øː/ \\
ä & /ɛ/ & \H{a} & /ɛː/ \\
\end{tabular}
\end{center}

Orthographic diphthongs may be monophthongs or diphthongs:

\begin{center}
\begin{tabular}{cc|cc}
ei & /iː/ & eï & /eɪ/ \\
eu & /øː/ & eü & /eʊ/ \\
ai & /ɛː/ & aï & /aɪ/ \\
au & /oː/ & aü & /aʊ/  \\ 
\end{tabular}
\end{center}

The weak vowel ⟨ę⟩ is articulated as a mid-central vowel /ə/, or may be \emph{elided} when elision does not produce too many obstruents in succession; when it is not ellided, it is not necessarily shorter than short vowels.

\subsection{Consonants}

\subsubsection{Sonorants}

Four ``sonorant'' consonants are important to correct Þeudisk-H orthography.

These are: ⟨j⟩, ⟨w⟩, ⟨m⟩, ⟨n⟩, ⟨l⟩, and ⟨r⟩.

The sonorants affect the spelling of the fricatives.

\subsubsection{Fricatives}

There are three voiced-unvoiced pairs of fricatives in Þeudisk-H:

\begin{itemize}
\item ⟨v⟩ and ⟨f⟩,
\item ⟨z⟩ and ⟨s⟩,
\item ⟨ð⟩ and ⟨þ⟩.
\end{itemize}

Although their standard pronunciations are respectively voiced and unvoiced,
their actual realisations may be voiced or unvoiced according to dialectal
variations, or euphony.

Given that ``vowels-or-sonorants'' is the union set of vowels and sonorants, and that the following rules are ``evaluated'' \textit{after} any suffixes and enclitics are applied to the word, the fricatives are spelled as according to:

\begin{itemize}
\item Start of a morpheme
    \begin{itemize}
    \item Unvoiced: always e.g. \textit{\textbf{þ}örv.an} (``to need'').
    \end{itemize}
\item End of a morpheme
    \begin{itemize}
    \item Voiced: if between two vowels-or-sonorants e.g. \textit{þör\textbf{v}.em} (``we need'').
    \item Unvoiced: elsewhere e.g. \textit{þu þör\textbf{f}.des} (``thou neededst'').
    \end{itemize}
\end{itemize}
\section{Nouns}

\subsection{Noun inflection paradigm}

Nouns declinate for nominative-accusative and genitive, and singular and plural.

r-stems:

\begin{center}
\begin{tabular}{ll}
Singular & Plural \\
\hline
bróþär  & brőþäriiu \\
fádär   & fa̋där \\
dohtär  & döhtär \\
módär   & mődär \\
swéstär & swéstärĕn \\
\end{tabular}
\end{center}

The genitive is formed by attaching -s at the end of a noun after declining it for its number (or -es after ⟨z⟩ or ⟨s⟩). For example:

\begin{center}
\begin{tabular}{l|l}
þe kat > þes kats     & þej käten > þer kätens  \\
þe hund > þes hunds   & þej hünden > þer hündens \\
þe wáter > þes wáters & þej w\H{a}t'ren > þer w\H{a}t'rens \\
þe leuht > þes leuhts & þej leühten > þer leühtens \\
þe aug > þes augs     & þej aügen > þer aügens \\
\end{tabular}
\end{center}

\section{Adjectives}

Adjectives inflect according to their grammatical function, number, and case.

In the \textbf{predicative} form, the adjective does not inflect.

In the \textbf{attributive} form, a suffix is attached to the adjective
according to case and number. An attributive adjective usually \textit{precedes} the
noun it modifies.

When the adjective \textit{follows} its noun, it is prefixed with ``ge-''. This belongs to a poetic register.

Weak inflection paradigm:

\begin{center}
\begin{tabular}{|c|cc|}
\hline
& Singular & Plural \\
\hline
Nominative & -e (see note)  & -en \\
Accusative & -en & -en \\
Genitive   & -es & -en \\
\hline
\end{tabular}
\end{center}

Strong inflection paradigm:

\begin{center}
\begin{tabular}{|c|cc|}
\hline
& Singular & Plural \\
\hline
Nominative & -e (see note)  & -ej \\
Accusative & -en & -em \\
Genitive   & -es & -er \\
\hline
\end{tabular}
\end{center}

Note: in the weak and strong inflection paradigms, the nominative final -ĕ is dropped when the adjective succeeds the noun i.e. \textit{þe hongrigĕ hond} but \textit{þe hond g'hongrig} (``the hungry dog'' and ``the dog a-hungry'').

Examples:

\begin{itemize}
\item Þe kat is hongrig (predicative).
\item Þe hongrige kat (before, sg., nom.).
\item Þe kat g'hongrig ëtet (after, sg., nom.).
\item Þe hongrigen wolf ëtet þen kat (before, sg., acc.).
\item Þe wolf gehongrig þen kat ëtet (after, sg., acc.).
\item Þej hongrigej hönd
\end{itemize}

\subsection{Comparative and superlative}

The comparative degree is derived by umlauting the last non-weak vowel and adding the suffix \textit{-ir}. Comparative adjectives are declined as regular adjectives. Compared objects are in nominative case, and reference objects in the genitive case.

The superlative degree is derived by umlauting the last non-weak vowel and adding the suffix \textit{-ist}. Superlative adjectives are declined as regular adjectives.

Examples:

\begin{itemize}
\item Þe mús is klaïnir þes kuwes, \textit{or} Þe mús þes kuwes klaïnir is (\textit{The mouse is smaller than the cow}).
\item Þe bláw hwal is þe graütiste fisk in þen mór (\textit{The blue whale is the biggest fish in the ocean}).
\item Þe bl\H{a}wiste hwal þes bl\H{a}wistes hájs bl\H{a}wir is. (\textit{The bluest whale is bluer than then bluest shark}).
\item Þej bl\H{a}wistej hwäl þer bl\H{a}wister h\H{a}js bl\H{a}wir sűn. (\textit{The bluest whale is bluer than then bluest shark}).
\end{itemize}

Irregular comparatives and superlatives include:

\begin{itemize}
\item gód > bätir > bäst
\item feul > maïr > maïst
\end{itemize}

\section{Pro-forms}

First and second person pronouns:

\begin{center}
\begin{tabular}{lllll}
& Nom. & Nom. (stressed) & Accusative & Genitive \\
\hline
Sg. 1st & ek  & ik  & meh & mein  \\
Sg. 2nd & þuw & þuj & þeh & þein \\
Pl. 1st & wir & wij & uns & unzer \\
Pl. 2nd & jir & jij & juh & jirwer \\
\end{tabular}
\end{center}

The nominative has unmarked forms and stressed forms. Their distinction is the same found in modern Dutch.

The third person pronouns:

\begin{center}
\begin{tabular}{llllll}
& Nom. & Nom. (stressed) & Accusative & Genitive & Meaning \\
\hline
Sg. & er  & ij  & em  & es  & someone, "dummy pronoun" \\
Sg. & her & hij & hem & hes & this person \\
Sg. & ser & sij & sem & ses & that person \\
Pl. & þej & þij & þem & þer & they \\ 
\end{tabular}
\end{center}

All third person pronouns have the reflexive accusative \textbf{\textit{sih}} and the reflexive gentivie \textbf{\textit{sein}}.

Pro-forms take the place of nouns, adverbs, or adjectives.

The pro-things are:

\begin{center}
\begin{tabular}{llll}
Nominative & Accusative & Genitive & Meaning \\
\hline
hwas       & hwat  & hwazes  & what  \\
hjas       & hjat  & hjazes  & this/these  \\
þis / þize & þit   & þizes   & this  \\
þizej      & þizem & þizer   & these \\
þas / þaze & þat   & þazes   & that  \\
þasej      & þazem & þazer   & those \\
jen        & jent  & jënes   & that over there \\
jënej      & jënem & jëner   & those over there \\
\hline
iht       & ihtet       & ihtes       & something \\
niht      & nihtet      & nihtes      & nothing \\
all'iht   & all'ihet    & all'ihtes   & everything \\
ainag'iht & ainag'ihten & ainag'ihtes & anything \\
\end{tabular}
\end{center}

Note that for þi-, þa-, je- demostratives, the endings follow a regular pattern. The epenthetic -e appears in adjective form, but not as pronoun.

The pro-persons are:

\begin{center}
\begin{tabular}{lllll}
Nominative & Accusative & Genitive & Meaning \\
\hline
hwës      & hwëm           & hwëzes       & who       \\
jëman     & jëmanen        & jëmanes      & somebody  \\
njëman    & njëmanen       & njëmanes     & nobody    \\
all'man   & all'manen      & allmanes     & everybody \\
ainag'man & ainag'manen    & ainag'manes  & anyone    \\
\end{tabular}
\end{center}

The pro-forms of alternative are:

\begin{center}
\begin{tabular}{lllll}
Nominative & Accusative & Genitive & Meaning \\
\hline
ëlk[e]   & ëlket   & ëlkes   & each \\
hwëlk[e] & hwëlket & hwëlkes & which \\
hjëlk[e] & hjëlket & hjëlkes  & this one \\
þëlk[e]  & þëlket  & þëlkes  & that one \\
sëlk[e]  & sëlket  & sëlkes  & such one \\
nëlk[e]  & nëlket  & nëlkes  & none \\
\end{tabular}
\end{center}

The pro-time, pro-occurrence:

\begin{center}
\begin{tabular}{lll}
Pro-form & Meaning \\
\hline
hwan & when \\
hjan & now, in this moment \\
þan & then (in that moment) \\
\hline
oit  & sometime, ever \\
noit & never \\
\end{tabular}
\end{center}

The pro-place:

\begin{center}
\begin{tabular}{lll}
Pro-form & Meaning \\
\hline
hwar       & where \\
hjar       & here \\
þar        & there \\
janer      & over there, yonder \\
ainag'hwar & anywhere \\
all'hwar   & everywhere \\
n'hwar     & nowhere \\
\end{tabular}
\end{center}

The pro-manner:

\begin{center}
\begin{tabular}{lll}
Pro-form & Meaning \\
\hline
hwus   & how \\
hjus   & in this way  \\
þus    & in that way, thus \\
sus    & in such way, so \\
all'us & everyhow \\
n'hwus & in no way \\
\end{tabular}
\end{center}

The pro-duration:

\begin{center}
\begin{tabular}{lll}
Pro-form & Meaning \\
\hline
hwulang & how long  \\
hjulang & this long \\
þulang  & that long \\
sulang  & so long   \\
\end{tabular}
\end{center}

The pro-origins (place or moment of start) are:

\begin{center}
\begin{tabular}{lll}
Pro-form & Meaning \\
\hline
hwans & since when, from where, whence \\
hjans & since now, from here, hence \\
þans  & since then, from there, thence \\
sans  & since then, since \\
\end{tabular}
\end{center}
\section{Verbs}

Verbs can be conjugated for 3 persons, 2 numbers, and 2 morphological tenses (present and preterite). Other tenses and moods can be constructed with periphrasis (like ``shall do'', ``would do'', ``had done'').

Non-conjugated verb forms include:

\begin{itemize}
\item the present infinitive,
\item the past infinitive (``to have done''),
\item the active participle,
\item the passive participle.
\end{itemize}

There are two main types of verbs: \textbf{strong verbs} and \textbf{weak verbs}. They differ in the preterite construction. Strong verbs have vowel alternations. Weak verbs have a dental suffix.

There is a handful of \textbf{preterite-present verbs} and \textbf{some irregular verbs}.

Verb conjugations vary in tense, number, and person. For both weak and strong verbs, the present tense suffixes are:

\begin{center}
\begin{tabular}{llll}
Singular & Suffix & Plural & Suffix \\
\hline
1st. p. & -ew & 1st. p. & -em \\
2nd. p. & -es & 2nd. p. & -ej \\
3rd. p. & -et & 3rd. p. & -en \\
\end{tabular}
\end{center}

In the second and third person singular, the vowel of the suffix may be ellided in speech, but not in writing.

The preterites of strong verbs have the following suffixes:

\begin{center}
\begin{tabular}{llll}
Singular & Suffix & Plural & Suffix \\
\hline
1st. p. & -   & 1st. p. & -em \\
2nd. p. & -ĕs & 2nd. p. & -ej \\
3rd. p. & -   & 3rd. p. & -en \\
\end{tabular}
\end{center}

The preterites of weak verbs have the following suffixes:

\begin{center}
\begin{tabular}{llll}
Singular & Suffix & Plural & Suffix \\
\hline
1st. p & -ed  & 1st. & -dem \\
2nd. p & -des & 2nd. & -dej \\
3rd. p & -ed  & 3rd. & -den \\
\end{tabular}
\end{center}

\subsection{Strong verbs paradigm}

Verbs are verbs which undergo ablaut to mark preterite or past participle.  They have 3 possible stems:

\begin{itemize}
\item present stem, used in present conjugations, present infinitive, and present participle;
\item preterite stem, used in the preterite conjugations, preterite infinitive;
\item past participle stem, used in the past participle, and some substantivations.
\end{itemize}

The following table summarises the ablaut vowel alternations in those stems:

\begin{center}
\begin{tabular}{lllll}
Class & Present & Preterite & Past part. & Example \\
\hline
1   & ei & ai & i  & leiþan  \\
2   & eu & au & u  & freuzan \\
34i & i  & a  & u  & brinnan \\
34e & e  & a  & o  & werðan  \\
5   & e  & a  & e  & nezan   \\
5i  & i  & a  & e  & sitan   \\
6   & a  & ú  & a  & faran   \\
7a  & ai & ei & ai & haitan  \\
7b  & au & eu & au & hlaupan \\
7c  & a  & e  & a  & hanhan  \\
7d  & á  & í  & á  & rádan   \\
7e  & ó  & í  & ó  & grówan  \\
\end{tabular}
\end{center}

There are some consonant alternations in the final consonants of strong verbs in the past participle paradigm:

\begin{center}
\begin{tabular}{ll}
Alternation & Example \\
\hline
v > b & sweivan > swiben \\
ð > d & werðan > worden \\
z > r & freuzan > fruren \\
\end{tabular}
\end{center}

Additionally, because of orthographic and euphonic rules, ⟨v⟩, ⟨ð⟩, and ⟨z⟩ alternate regularly with ⟨f⟩, ⟨þ⟩, and ⟨s⟩ in some phonological contexts. This is not unique to the conjugation paradigm.

For example, for the verb \textit{hwervan} (``to throw''):

\begin{itemize}
\item He hwërft þen kat (\textit{He throws the cat}).
\item We hwërvem þen kat (\textit{We throw the cat}).
\item Þen kat was hworben (\textit{The cat was thrown}).
\end{itemize}

\subsection{Weak verb paradigm}

Irregular verbs have two possible stems:

\begin{itemize}
\item the present stem, for present conjugations, present infinitive, present participle;
\item the past preterite stem, for past conjugations, past infinitive, and past participle.
\end{itemize}

Class 1 - causative
-an

Class 2 - denominals

fiskon fiskod-

Class 3 - factitive (stative are irregular)
-aan

Class 4 - inchoative (go into state X)
waknan

\subsection{Preterite-present verbs}

These verbs include preterite-present verbs in Proto-Germanic, plus a cognate of ``will''.

They are different from other verbs in that the present stem is different for the singular and plural. The present infinitive shows the present plural stem. They

The preterite-present verbs are:

\begin{center}
\begin{tabular}{ll}
Present inf. & Meaning \\
\hline
witan  & to know \\
aigan  & to own \\
dügan  & to be fit or useful for \\
künnan & to be able to \\
ünnan  & to allow \\
þörvan & to need \\
dörran & to dare \\
skölan & \begin{tabular}{l}(present) future tense\\(preterite) cond. mood\end{tabular} \\
münan  & to believe \\
nögan  & to be enough \\
mögan  & to be allowed to \\
mötan  & to have to \\
ögan   & to be scared of \\
wölan  & to want \\
\end{tabular}
\end{center}

The following table includes the present singular stem, the present plural stem, the preterite stem.

\begin{center}
\begin{tabular}{clll}
Present. sg. & Present. pl. & Preterite \\
\hline
wait- & wit-  & wist-  \\
aih-  & aig-  & aiht-  \\
daug- & deig- & duhd-  \\
kann- & kunn- & kunnd- \\
ann-  & ünn-  & unnd-  \\
þarv- & þörv- & þorft- \\
darz- & dörr- & dorst- \\
skal- & sköl- & skold- \\
man-  & mün-  & mund-  \\
nag-  & nüg-  & nuht-  \\
mag-  & mög-  & moht-  \\
mut-  & müt-  & must-  \\
ag-   & og-   & oht-   \\
wil-  & wöl-  & wold-  \\
\end{tabular}
\end{center}

The following is the conjugation paradigm for all preterite-present verbs. Notice that the present suffix corresponds to the preterite suffix of strong verbs, and the preterite suffix corresponds to the preterite suffix of weak verbs.

\begin{center}
\begin{tabular}{lll}
Person & Present suffix & Preterite suffix \\
\hline
1 sg. &     & -ed \\
2 sg. & -es & -eds \\
3 sg. & -   & -ed \\
1 pl. & -em & -dem \\
2 pl. & -ed & -ded \\
3 pl. & -en & -den \\
\end{tabular}
\end{center}

\subsection{Irregular verbs}

Irregular verbs include some irregular weak verbs, and some suppletive paradigms.

The stems of the \textbf{irregular weak verbs} are:

\begin{center}
\begin{tabular}{lll}
Present infinitive & Past infinitive & Meaning \\
\hline
bringan & brahtan & to bring \\
brukan  & bruhtan & to use   \\
buwan   & buwdan  & to dwell (not irregular!) \\
\hline
bügan  & buhtan  & to buy   \\
sőkan  & sóhtan  & to seek  \\
þenkan & þahtan  & to think \\
þonkan & þuhtan  & to seem  \\
würkan & wurhtan & to work  \\
\hline
hevan & haddan & to have \\
livan & lifdan & to be alive (not irregular!) \\
hügan & hugdan & to remember \\
\end{tabular}
\end{center}

Just like with regular weak verbs:

\begin{itemize}
\item the present infinitive, present participle, and present conjugations share the same stem,
\item the past infinitive, past participle, and preterite conjugations share the same stem.
\end{itemize}

The rest of irregular verbs are: \textbf{\emph{ween}}, \textbf{\emph{gaan}}, \textbf{\emph{staan}}, \textbf{\emph{doon}}.

\textbf{\emph{ween}} is the most irregular verb in Þeudisk-H. The full conjugation paradigm is given here:

\begin{center}
\begin{tabular}{llll}
& Present & Past & Imperative \\
\hline
1st. sg. & am  & war   &     \\
2nd. sg. & irs & wars  & wés \\
3rd. sg. & is  & war   &     \\
1st. pl. & sűm & warem &     \\
2nd. pl. & sűj & warej &     \\
3rd. pl. & sűn & waren &     \\
\hline
Infinitive & weĕn  & warĕn  \\
Participle & weĕnd & wesĕn \\
\end{tabular}
\end{center}

The full conjugation paradigm for \textbf{\emph{gaan}}:

\begin{center}
\begin{tabular}{llll}
& Present & Past & Imperative \\
\hline
1st. sg. & gáw & ging  & \\
2nd. sg. & gás & gings & gá \\
3rd. sg. & gát & ging  & \\
1st. pl. & gám & gingĕm & \\
2nd. pl. & gáj & gingĕj & \\
3rd. pl. & gán & gingĕn & \\
\hline
Infinitive & gaan & gingan \\
Participle & gaand & gangĕn \\
\end{tabular}
\end{center}

The full conjugation paradigm for \textbf{\emph{staan}}:

\begin{center}
\begin{tabular}{llll}
& Present & Past & Imperative \\
\hline
1st. sg. & standĕw & stód   & \\
2nd. sg. & standĕs & stóds  & stá \\
3rd. sg. & standĕt & stód   & \\
1st. pl. & standĕm & stódĕm & \\
2nd. pl. & standĕj & stódĕj & \\
3rd. pl. & standĕn & stódĕn & \\
\hline
Infinitive & standan & stodan \\
Participle & stanand & standĕn \\
\end{tabular}
\end{center}

The full conjugation paradigm for \textbf{\emph{doon}}:

\begin{center}
\begin{tabular}{llll}
& Present & Past & Imperative \\
\hline
1st. sg. & dów & dád  & \\
2nd. sg. & dós & dáds & dó \\
3rd. sg. & dót & dád  & \\
1st. pl. & dóm & dádĕm & \\
2nd. pl. & dój & dádĕj & \\
3rd. pl. & dón & dádĕn & \\
\hline
Infinitive & doon & dádan \\
Participle & doond & daan  \\
\end{tabular}
\end{center}

\section{Adpositions}

\subsection{Adpositions of place and time}

Adpositions for place and time can derivate into several forms.

They can be static prepositions that generally show where or when the referred
noun is; whereas directional prepositions show the direction towards which some
movement occurs.

For example:

\begin{itemize}
\item Þe kat hlaupt in þen hús. (``The cat walks inside the house'').
\item Þe kat hlaupt intil þen hús. (``The cat walks into the house'').
\end{itemize}

The predicative and attributive forms function as adjectives or adverbs, for example:

\begin{itemize}
\item Þe aft're mál (``The later meal'').
\item Þe mál g'aft're (``The meal a-later'').
\end{itemize}

This is a list of place and time adpositions, and their derived terms.

\begin{center}
\begin{tabular}{lllll}
Meaning & Static & Direction & Predicative & Attributive \\
\hline
\textit{before}  & för     & förtil     & b'fören    & före     \\
\textit{behind}  & hinder  & hind'rtil  & b'hind'ren & hindre   \\
\textit{after}   & after   & -          & afteren    & aftre    \\
\textit{in}      & in      & intil      & binen      & bin're   \\
\textit{out}     & ut      & uttil      & buten      & but're   \\
\textit{below}   & niðer   & (nið'rtil) & b'nið'ren  & nið're   \\
\textit{above}   & över    & (öv'rtil)  & böv'ren    & öv're    \\
\textit{on}      & op      & optil      & open       & opre     \\
\textit{under}   & onder   & ond'rtil   & ond'ren    & ond're   \\
\textit{between} & twisken & twisk'ntil & twisken    & twisk'ne \\
\end{tabular}
\end{center}

\textit{för} and its variations refer to both ``before'' in time or ``in front of''.

\textit{optil} and \textit{ondr'til} are similar in meaning to \textit{öv'rtil} and \textit{nið'rtil}, but the former are preferred to the latter.

\subsection{Other adpositions}

\begin{center}
\begin{tabular}{lll}
Preposition & Case & Use \\
\hline
an   & acc. & introduces indirect object \\
bi   & nom. & indicates company and location \\
meþ  & nom. & indicates company \\
til  & acc. & direction of movement \\
af   & nom. & away from \\
hwil & nom. & during \\
om   & nom. & about \\
\end{tabular}
\end{center}

\section{Conjunctions}

\begin{itemize}
\item ond
\end{itemize}

Many conjunctions derive from prepositions, for example:

\begin{itemize}
\item hweil þat - while
\item om þat - because
\end{itemize}


\end{document}
