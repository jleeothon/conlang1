\section{Phonology and orthography}

The Þeudisc-H alphabet is based on the Latin alphabet. Alternate spelling systems are acceptable but not included in the scope of this document.

\subsection{Vowels}

Þeudisk-K vowels be summarised as:

\begin{itemize}
\item short monophthongs,
\item a weak vowel vowel,
\item long vowels and diphthongs.
\end{itemize}

Vowels may be ``plain'' or ``umlauted''. Plain vowels appear anywhere, and umlauted variants occur in certain derivation paradigms.

Short vowels and orthographic monophthongs:

\begin{center}
\begin{tabular}{cc|cc}
\multicolumn{2}{c}{Short} &
\multicolumn{2}{c}{Long} \\
\hline
i & /ɪ/ & í & /iː/ \\
e & /ɛ/ & é & /eː/ \\
u & /ʊ/ & ú & /uː/ \\
o & /ɔ/ & ó & /oː/ \\
a & /ɑ/ & á & /aː/ \\
ü & /ʏ/ & \H{u} & /yː/ \\
ö & /œ/ & \H{o} & /øː/ \\
ä & /ɛ/ & \H{a} & /ɛː/ \\
\end{tabular}
\end{center}

Orthographic diphthongs may be monophthongs or diphthongs:

\begin{center}
\begin{tabular}{cc|cc}
ei & /iː/ & eï & /eɪ/ \\
eu & /øː/ & eü & /eʊ/ \\
ai & /ɛː/ & aï & /aɪ/ \\
au & /oː/ & aü & /aʊ/  \\
\end{tabular}
\end{center}

The weak vowel ⟨ę⟩ is articulated as a mid-central vowel /ə/, or may be \emph{elided} when elision does not produce too many obstruents in succession; when it is not ellided, it is not necessarily shorter than short vowels.

\subsection{Consonants}

\subsubsection{Sonorants}

Four ``sonorant'' consonants are important to correct Þeudisk-H orthography.

These are: ⟨j⟩, ⟨w⟩, ⟨m⟩, ⟨n⟩, ⟨l⟩, and ⟨r⟩.

The sonorants affect the spelling of the fricatives.

\subsubsection{Fricatives}

There are three voiced-unvoiced pairs of fricatives in Þeudisk-H:

\begin{itemize}
\item ⟨v⟩ and ⟨f⟩,
\item ⟨z⟩ and ⟨s⟩,
\item ⟨ð⟩ and ⟨þ⟩.
\end{itemize}

Although their standard pronunciations are respectively voiced and unvoiced,
their actual realisations may be voiced or unvoiced according to dialectal
variations, or euphony.

Given that ``vowels-or-sonorants'' is the union set of vowels and sonorants, and that the following rules are ``evaluated'' \textit{after} any suffixes and enclitics are applied to the word, the fricatives are spelled as according to:

\begin{itemize}
\item Start of a morpheme
    \begin{itemize}
    \item Unvoiced: always e.g. \textit{\textbf{þ}örv.an} (``to need'').
    \end{itemize}
\item End of a morpheme
    \begin{itemize}
    \item Voiced: if between two vowels-or-sonorants e.g. \textit{þör\textbf{v}.em} (``we need'').
    \item Unvoiced: elsewhere e.g. \textit{þu þör\textbf{f}.des} (``thou neededst'').
    \end{itemize}
\end{itemize}
