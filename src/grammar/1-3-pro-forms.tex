\section{Pro-forms}

First and second person pronouns:

\begin{center}
\begin{tabular}{lllll}
& Nom. & Nom. (stressed) & Accusative & Genitive \\
\hline
Sg. 1st & ek  & ik  & meh & mein  \\
Sg. 2nd & þuw & þuj & þeh & þein \\
Pl. 1st & wir & wij & uns & unzer \\
Pl. 2nd & jir & jij & juh & jirwer \\
\end{tabular}
\end{center}

The nominative has unmarked forms and stressed forms. Their distinction is the same found in modern Dutch.

The third person pronouns:

\begin{center}
\begin{tabular}{llllll}
& Nom. & Nom. (stressed) & Accusative & Genitive & Meaning \\
\hline
Sg. & er  & ij  & em  & es  & someone, "dummy pronoun" \\
Sg. & her & hij & hem & hes & this person \\
Sg. & ser & sij & sem & ses & that person \\
Pl. & þej & þij & þem & þer & they \\ 
\end{tabular}
\end{center}

All third person pronouns have the reflexive accusative \textbf{\textit{sih}} and the reflexive gentivie \textbf{\textit{sein}}.

Pro-forms take the place of nouns, adverbs, or adjectives.

The pro-things are:

\begin{center}
\begin{tabular}{llll}
Nominative & Accusative & Genitive & Meaning \\
\hline
hwas       & hwat  & hwazes  & what  \\
hjas       & hjat  & hjazes  & this/these  \\
þis / þize & þit   & þizes   & this  \\
þizej      & þizem & þizer   & these \\
þas / þaze & þat   & þazes   & that  \\
þasej      & þazem & þazer   & those \\
jen        & jent  & jënes   & that over there \\
jënej      & jënem & jëner   & those over there \\
\hline
iht       & ihtet       & ihtes       & something \\
niht      & nihtet      & nihtes      & nothing \\
all'iht   & all'ihet    & all'ihtes   & everything \\
ainag'iht & ainag'ihten & ainag'ihtes & anything \\
\end{tabular}
\end{center}

Note that for þi-, þa-, je- demostratives, the endings follow a regular pattern. The epenthetic -e appears in adjective form, but not as pronoun.

The pro-persons are:

\begin{center}
\begin{tabular}{lllll}
Nominative & Accusative & Genitive & Meaning \\
\hline
hwës      & hwëm           & hwëzes       & who       \\
jëman     & jëmanen        & jëmanes      & somebody  \\
njëman    & njëmanen       & njëmanes     & nobody    \\
all'man   & all'manen      & allmanes     & everybody \\
ainag'man & ainag'manen    & ainag'manes  & anyone    \\
\end{tabular}
\end{center}

The pro-forms of alternative are:

\begin{center}
\begin{tabular}{lllll}
Nominative & Accusative & Genitive & Meaning \\
\hline
ëlk[e]   & ëlket   & ëlkes   & each \\
hwëlk[e] & hwëlket & hwëlkes & which \\
hjëlk[e] & hjëlket & hjëlkes  & this one \\
þëlk[e]  & þëlket  & þëlkes  & that one \\
sëlk[e]  & sëlket  & sëlkes  & such one \\
nëlk[e]  & nëlket  & nëlkes  & none \\
\end{tabular}
\end{center}

The pro-time, pro-occurrence:

\begin{center}
\begin{tabular}{lll}
Pro-form & Meaning \\
\hline
hwan & when \\
hjan & now, in this moment \\
þan & then (in that moment) \\
\hline
oit  & sometime, ever \\
noit & never \\
\end{tabular}
\end{center}

The pro-place:

\begin{center}
\begin{tabular}{lll}
Pro-form & Meaning \\
\hline
hwar       & where \\
hjar       & here \\
þar        & there \\
janer      & over there, yonder \\
ainag'hwar & anywhere \\
all'hwar   & everywhere \\
n'hwar     & nowhere \\
\end{tabular}
\end{center}

The pro-manner:

\begin{center}
\begin{tabular}{lll}
Pro-form & Meaning \\
\hline
hwus   & how \\
hjus   & in this way  \\
þus    & in that way, thus \\
sus    & in such way, so \\
all'us & everyhow \\
n'hwus & in no way \\
\end{tabular}
\end{center}

The pro-duration:

\begin{center}
\begin{tabular}{lll}
Pro-form & Meaning \\
\hline
hwulang & how long  \\
hjulang & this long \\
þulang  & that long \\
sulang  & so long   \\
\end{tabular}
\end{center}

The pro-origins (place or moment of start) are:

\begin{center}
\begin{tabular}{lll}
Pro-form & Meaning \\
\hline
hwans & since when, from where, whence \\
hjans & since now, from here, hence \\
þans  & since then, from there, thence \\
sans  & since then, since \\
\end{tabular}
\end{center}