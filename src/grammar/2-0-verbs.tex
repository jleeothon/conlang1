\section{Verbs}

Verbs can be conjugated for 3 persons, 2 numbers, and 2 morphological tenses (present and preterite). Other tenses and moods can be constructed with periphrasis (like ``shall do'', ``would do'', ``had done'').

Non-conjugated verb forms include:

\begin{itemize}
\item the present infinitive,
\item the past infinitive (``to have done''),
\item the active participle,
\item the passive participle.
\end{itemize}

There are two main types of verbs: \textbf{strong verbs} and \textbf{weak verbs}. They differ in the preterite construction. Strong verbs have vowel alternations. Weak verbs have a dental suffix.

There is a handful of \textbf{preterite-present verbs} and \textbf{some irregular verbs}.

Verb conjugations vary in tense, number, and person. For both weak and strong verbs, the present tense suffixes are:

\begin{center}
\begin{tabular}{llll}
Singular & Suffix & Plural & Suffix \\
\hline
1st. p. & -ew & 1st. p. & -em \\
2nd. p. & -es & 2nd. p. & -ej \\
3rd. p. & -et & 3rd. p. & -en \\
\end{tabular}
\end{center}

In the second and third person singular, the vowel of the suffix may be ellided in speech, but not in writing.

The preterites of strong verbs have the following suffixes:

\begin{center}
\begin{tabular}{llll}
Singular & Suffix & Plural & Suffix \\
\hline
1st. p. & -   & 1st. p. & -em \\
2nd. p. & -ĕs & 2nd. p. & -ej \\
3rd. p. & -   & 3rd. p. & -en \\
\end{tabular}
\end{center}

The preterites of weak verbs have the following suffixes:

\begin{center}
\begin{tabular}{llll}
Singular & Suffix & Plural & Suffix \\
\hline
1st. p & -ed  & 1st. & -dem \\
2nd. p & -des & 2nd. & -dej \\
3rd. p & -ed  & 3rd. & -den \\
\end{tabular}
\end{center}

\subsection{Strong verbs paradigm}

Verbs are verbs which undergo ablaut to mark preterite or past participle.  They have 3 possible stems:

\begin{itemize}
\item present stem, used in present conjugations, present infinitive, and present participle;
\item preterite stem, used in the preterite conjugations, preterite infinitive;
\item past participle stem, used in the past participle, and some substantivations.
\end{itemize}

The following table summarises the ablaut vowel alternations in those stems:

\begin{center}
\begin{tabular}{lllll}
Class & Present & Preterite & Past part. & Example \\
\hline
1   & ei & ai & i  & leiþan  \\
2   & eu & au & u  & freuzan \\
34i & i  & a  & u  & brinnan \\
34e & e  & a  & o  & werðan  \\
5   & e  & a  & e  & nezan   \\
5i  & i  & a  & e  & sitan   \\
6   & a  & ú  & a  & faran   \\
7a  & ai & ei & ai & haitan  \\
7b  & au & eu & au & hlaupan \\
7c  & a  & e  & a  & hanhan  \\
7d  & á  & í  & á  & rádan   \\
7e  & ó  & í  & ó  & grówan  \\
\end{tabular}
\end{center}

There are some consonant alternations in the final consonants of strong verbs in the past participle paradigm:

\begin{center}
\begin{tabular}{ll}
Alternation & Example \\
\hline
v > b & sweivan > swiben \\
ð > d & werðan > worden \\
z > r & freuzan > fruren \\
\end{tabular}
\end{center}

Additionally, because of orthographic and euphonic rules, ⟨v⟩, ⟨ð⟩, and ⟨z⟩ alternate regularly with ⟨f⟩, ⟨þ⟩, and ⟨s⟩ in some phonological contexts. This is not unique to the conjugation paradigm.

For example, for the verb \textit{hwervan} (``to throw''):

\begin{itemize}
\item He hwërft þen kat (\textit{He throws the cat}).
\item We hwërvem þen kat (\textit{We throw the cat}).
\item Þen kat was hworben (\textit{The cat was thrown}).
\end{itemize}

\subsection{Weak verb paradigm}

Irregular verbs have two possible stems:

\begin{itemize}
\item the present stem, for present conjugations, present infinitive, present participle;
\item the past preterite stem, for past conjugations, past infinitive, and past participle.
\end{itemize}

Class 1 - causative
-an

Class 2 - denominals

fiskon fiskod-

Class 3 - factitive (stative are irregular)
-aan

Class 4 - inchoative (go into state X)
waknan

\subsection{Preterite-present verbs}

These verbs include preterite-present verbs in Proto-Germanic, plus a cognate of ``will''.

They are different from other verbs in that the present stem is different for the singular and plural. The present infinitive shows the present plural stem. They

The preterite-present verbs are:

\begin{center}
\begin{tabular}{ll}
Present inf. & Meaning \\
\hline
witan  & to know \\
aigan  & to own \\
dügan  & to be fit or useful for \\
künnan & to be able to \\
ünnan  & to allow \\
þörvan & to need \\
dörran & to dare \\
skölan & \begin{tabular}{l}(present) future tense\\(preterite) cond. mood\end{tabular} \\
münan  & to believe \\
nögan  & to be enough \\
mögan  & to be allowed to \\
mötan  & to have to \\
ögan   & to be scared of \\
wölan  & to want \\
\end{tabular}
\end{center}

The following table includes the present singular stem, the present plural stem, the preterite stem.

\begin{center}
\begin{tabular}{clll}
Present. sg. & Present. pl. & Preterite \\
\hline
wait- & wit-  & wist-  \\ 
aih-  & aig-  & aiht-  \\
daug- & düg-  & duhd-  \\ 
kann  & künn- & kunnd- \\
ann-  & ünn-  & unnd-  \\
þarv- & þörv- & þorft- \\
darz- & dörr- & dorst- \\
skal- & sköl- & skold- \\
man-  & mün-  & mund-  \\
nag-  & nüg-  & nuht-  \\
mag-  & mög-  & moht-  \\
mut-  & müt-  & must-  \\
ag-   & og-   & oht-   \\
wil-  & wöl-  & wold-  \\
\end{tabular}
\end{center}

The following is the conjugation paradigm for all preterite-present verbs. Notice that the present suffix corresponds to the preterite suffix of strong verbs, and the preterite suffix corresponds to the preterite suffix of weak verbs.

\begin{center}
\begin{tabular}{lll}
Person & Present suffix & Preterite suffix \\
\hline
1 sg. &     & -ed \\
2 sg. & -es & -eds \\
3 sg. & -   & -ed \\
1 pl. & -em & -dem \\
2 pl. & -ed & -ded \\
3 pl. & -en & -den \\
\end{tabular}
\end{center}

\subsection{Irregular verbs}

Irregular verbs include some irregular weak verbs, and some suppletive paradigms.

The stems of the \textbf{irregular weak verbs} are:

\begin{center}
\begin{tabular}{lll}
Present infinitive & Past infinitive & Meaning \\
\hline
bringan & brahtan & to bring \\
brukan  & bruhtan & to use   \\
búan    & búdan  & to dwell (not irregular!) \\
\hline
bügan  & buhtan  & to buy   \\
sőkan  & sóhtan  & to seek  \\
þänkan & þahtan  & to think \\
þünkan & þuhtan  & to seem  \\
würkan & wurhtan & to work  \\
\hline
hävan & haddan & to have \\
livan & lifdan & to be alive (not irregular!) \\
hügan & hugdan & to remember \\
\end{tabular}
\end{center}

Just like with regular weak verbs:

\begin{itemize}
\item the present infinitive, present participle, and present conjugations share the same stem,
\item the past infinitive, past participle, and preterite conjugations share the same stem.
\end{itemize}

The rest of irregular verbs are: \textbf{\emph{ween}}, \textbf{\emph{gaan}}, \textbf{\emph{staan}}, \textbf{\emph{doon}}. 

\textbf{\emph{ween}} is the most irregular verb in Þeudisk-H. The full conjugation paradigm is given here:

\begin{center}
\begin{tabular}{llll}
& Present & Past & Imperative \\
\hline
1st. sg. & am  & war   &     \\
2nd. sg. & irs & wars  & wés \\
3rd. sg. & is  & war   &     \\
1st. pl. & sűm & warem &     \\
2nd. pl. & sűj & warej &     \\
3rd. pl. & sűn & waren &     \\
\hline
Infinitive & weĕn  & warĕn  \\
Participle & weĕnd & wesĕn \\
\end{tabular}
\end{center}

The full conjugation paradigm for \textbf{\emph{gaan}}:

\begin{center}
\begin{tabular}{llll}
& Present & Past & Imperative \\
\hline
1st. sg. & gáw & ging  & \\
2nd. sg. & gás & gings & gá \\
3rd. sg. & gát & ging  & \\
1st. pl. & gám & gingĕm & \\
2nd. pl. & gáj & gingĕj & \\
3rd. pl. & gán & gingĕn & \\
\hline
Infinitive & gaan & gingan \\
Participle & gaand & gangĕn \\
\end{tabular}    
\end{center}

The full conjugation paradigm for \textbf{\emph{staan}}:

\begin{center}
\begin{tabular}{llll}
& Present & Past & Imperative \\
\hline
1st. sg. & stáw & stond   & \\
2nd. sg. & stás & stonds  & stá \\
3rd. sg. & stát & stond   & \\
1st. pl. & stám & stondĕm & \\
2nd. pl. & stáj & stondĕj & \\
3rd. pl. & stán & stondĕn & \\
\hline
Infinitive & staan & stondan \\
Participle & stanand & standĕn \\
\end{tabular}    
\end{center}

The full conjugation paradigm for \textbf{\emph{doon}}:

\begin{center}
\begin{tabular}{llll}
& Present & Past & Imperative \\
\hline
1st. sg. & dów & dád  & \\
2nd. sg. & dós & dáds & dó \\
3rd. sg. & dót & dád  & \\
1st. pl. & dóm & dádĕm & \\
2nd. pl. & dój & dádĕj & \\
3rd. pl. & dón & dádĕn & \\
\hline
Infinitive & doon & dádan \\
Participle & doond & daan \\
\end{tabular}    
\end{center}